\documentclass[a4paper,landscape, 11pt]{article}

\usepackage{multicol}

\usepackage[english]{babel}
\usepackage[utf8]{inputenc}
\usepackage{amsmath}
\usepackage{amssymb}
\usepackage{graphicx}
\usepackage{fancyhdr}
\usepackage{geometry}
\usepackage{mathptmx}
\usepackage{paralist}
\usepackage[compact]{titlesec}
\titlespacing{\section}{0pt}{2ex}{1ex}
\titlespacing{\subsection}{0pt}{1ex}{0ex}
\titlespacing{\subsubsection}{0pt}{0.5ex}{0ex}
\setlength\parindent{0pt}

\setlength{\parskip}{0cm}
\setlength{\parindent}{1em}
\geometry{
    left=0.5cm,
    right=0.5cm,
    top=1cm,
    bottom=1cm,
    bindingoffset=5mm
}


\author{Johann Wagner, Kai Seidensticker, Jasper Orschulko}


\newcommand{\limTo}[1]{ \lim\limits_{x \rightarrow #1}}
\newcommand{\limFromTo}[2]{ \lim\limits_{#2 \rightarrow #1}}

\newenvironment{alignTab}{$\begin{aligned}}{\end{aligned}$}

\title{Mathe II - Klausurhilfe}

\begin{document}
    \begin{multicols}{5}
    \begin{small}
    \section{Allgemeines}
    	\subsection{Gruppen}
    	Brauchen:
    	
    	\begin{compactitem}
    		\item {Assoziativität: $a\circ(b\circ c)=(a\circ b)\circ c$}
    		\item {neutrales Element: $\exists e\in G, a\circ e=a$}
    		\item {inverses Element: $\exists a^{-1}\in G, a\circ a^{-1}=e$}
    		\item {Kommutativität: $a\circ b=b\circ a$}
    	\end{compactitem}
    	\subsubsection{Untergruppen}
    	Brauchen:
    	
    	\begin{compactitem}   		
    		\item {$U\subseteq G$}
    		\item {U ist eine Gruppe (siehe Gruppe)}
    	\end{compactitem}
        \subsection{Potenzgesetze}
            \begin{multicols}{2}
            $a^m \cdot a^n = a^{m+n}$\\
            $a^n \cdot b^n = (ab)^n$\\
            $\frac{a^n}{a^m} = a^{n-m}$\\
            $\frac{a^n}{b^n} = \left(\frac{a}{b}\right)^n$\\
            $(a^n)^m = a^{mn}$\\
            $a^{-n} = \frac{1}{a^n}$\\
            $log_b(1) = 0$
            \end{multicols}
         \subsection{Logarithmus-Gesetze}
            \noindent
            $x = log_a(y) \Leftrightarrow y = a^x$\\
            $log(x) + log(y) = log(xy)$\\
            $log(x) - log(y) = log(\frac{x}{y})$\\
            $log_a(x) = \frac{log_b(x)}{log_b(a)}$  \\
            $log(u^r) = r \cdot ln(u)$
            \begin{multicols}{2}
            \noindent
            $ln(1) = 0$ \\
            $ln(e) = 1$ \\
            $ln(e^x) = x$ \\
            $e^{ln(x)} = x$
            \end{multicols}
        \subsection{Komplexe Zahlen}
        $(a + bi) \pm (c + di) = (a \pm c) + (c \pm d)i$\\
        $(a + bi) \cdot (c + di) = (ac - bd) + (ad + bc)i$\\
    
        $\displaystyle \frac{a + bi}{c + di} = \frac{ac + bd}{c^2 + d^2} + \frac{cb - ad}{c^2 + d^2}i$
        \subsection{Nett to know}
        \begin{compactitem}
        \item {$e^x$ hat keine Nullstelle}
        \item {$\log(\sqrt{x}) = \frac{\log(x)}{2}$}
        \item {$\log_a(b)=c \rightarrow a^c=b$}
        \item {$\log_x(b^x)=x$}
        \end{compactitem}
    \section{Permutationen}
    Ausführliche Darstellung: $\pi_1=\binom{1\ 2\ 3\ 4\ 5}{3\ 2\ 1\ 5\ 4}$
    
    Zyklische Darstellung:$\pi_1=(13)(45)$
    
    Komposition von Hinten anfangen:
    
    e.g. $\pi_1\circ\pi_2=(13)(45)\circ(2534)=(34521)$
    
    Bei $\pi^{-1}$: Zeilen vertauschen.
    
    \section{Determinanten, Eigenwerte, Eigenvektoren}
        \subsection{Determinante}
        A = $\left(
        \begin{matrix}
            2 & 3 \\
            5 & 1
        \end{matrix}
        \right)$ \hspace{2pt} 
        $det(A) = (2*1) - (5*3) = -13$ \\ \\
        3x3-Matrix:\\
        $det(B) = a_{11} a_{22} a_{33} + a_{12} a_{23} a_{31} + a_{13} a_{21} a_{32} - a_{13} a_{22} a_{31} - a_{11} a_{23} a_{32} - a_{12} a_{21} a_{33}$  
        
        \subsubsection{Gauss}
        Wenn eine Matrix in Dreiecksform ist, dann kann ihre Determinante an ihrer Hauptdiagonalen abgelesen werden. \\
        Wenn eine Zeile der Matrix multipliziert wird, muss die Determinante mit dem Kehrwert multipliziert werden.\\
        
        $G = \begin{pmatrix}
            1 & 2 & 3 \\
            0 & 2 & 1 \\
            0 & 0 & 3
        \end{pmatrix}
        \hspace{10pt}
        G' = \begin{pmatrix}
        
        2 & 4 & 6 \\
        0 & 2 & 1 \\
        0 & 0 & 3
        \end{pmatrix}
        $
        \\
        $\rightarrow det(G) = \frac{1}{2} det(G')$
        \subsection{Eigenwert}
        $det(A - E\lambda) = 
        \left(
        \begin{matrix}
            2-\lambda & 3 \\
            5 & 1-\lambda
        \end{matrix}
        \right) = (2-\lambda)(1-\lambda) - 5*3 = \lambda^2 - 3\lambda  + 2 - 15$\\
        pq-Formel und gib ihm: $-\frac{p}{2}\pm\sqrt{(\frac{p}{2})^2-q}$
        
        \subsection{Eigenvektor}
        Für jeden Eigenwert kann man einen Eigenvektor aufstellen, indem man folgende Gleichung löst: \\
        $ (A - \lambda_iE)*\vec{x} = 0$\\
        $\vec{x}$ ist der Eigenvektor.
        
        \subsection{Eigenraum}
        Die Kombination aller Eigenvektoren zu einer Matrix ergibt den Eigenraum.
        \subsection{Diagonalmatrix}
        $D=\begin{pmatrix}
        \lambda_1 & 0 & 0 \\
        0 & \lambda_2 & 0 \\
        0 & 0 & \lambda_3 
        \end{pmatrix}$
        \subsection{Ähnliche Matrizen}
        Zwei Matrizen $A, B$ sind sich ähnlich, wenn es eine Matrix $S$gibt, dass gilt: \\
        $S\cdot B = A \cdot S$
    \section{Folgen}
    $a_n < a_(n+1) \rightarrow$ streng monoton steigend. \\
    $a_n > a_(n+1) \rightarrow$ streng monoton fallend. \\ 
    \\ 
    Wenn es ein $C \in \mathbb{R}$ gibt, für das gilt, dass $a_n < C$, dann ist die Funktion nach Oben beschränkt. \\
    Wenn es ein $c \in \mathbb{R}$ gibt, für das gilt, dass $a_n > c$, dann ist die Funktion nach Unten beschränkt. \\
    Wenn es ein $m \in \mathbb{R}$ gibt, für das gilt, dass $ | a_n | < m$, dann ist die Funktion beschränkt.
        
        \subsection{Konvergenz und Divergenz}
            
            Eine Folge $a_n$ gilt als konvergent gegen $a \in \mathbb{R}$, wenn es für eine reelle Zahl $\epsilon > 0$ einen Folgeindex $n_0$ gibt, so dass die Folgeglieder $n>n_0$ in dem Intervall $\left[ a - \epsilon, a + \epsilon \right] $ liegt.\\
            $\rightarrow | a_n - a | < \epsilon$
            \\ \\
            Eine Folge heißt konvergent, wenn es einen eindeutigen Grenzwert gibt. \\
            Eine Folge heißt divergent, wenn sie nicht konvergiert. \\
            Eine Folge heißt bestimmt divergent, wenn für die Folge $a_n$ gilt, dass $\limFromTo{\infty}{n}\frac{1}{a_n} = 0$ gilt. In diesem Fall konvergiert die Folge gegen $\infty$.
        \subsection{l'Hospital}
            Hospital kann angewendet werden bei: \\
            $\frac{0}{0}, 0 \cdot \infty, \infty - \infty, \frac{\infty}{\infty}, 0^0, \infty^0, 1^\infty$\\
            Wenn kein Quotient gegeben ist, muss dieser durch Umformung erreicht werden - in Übung nicht behandelt.
            
            $\limFromTo{x0}{x} \frac{f'(x)}{g'(x)} = c \rightarrow \limFromTo{x0}{x} \frac{f(x)}{g(x)} = c$
            
    
    \section{Reihen}
        \subsection{Partialsumme}
            $\sum_{k=0}^{\infty} a_k = a_0 + a_1 + ... $
        \subsection{Konvergenzkriterium}
            \subsubsection{Leibnitz-Kriterium}
                \noindent$\sum_{k=0}^{\infty} (-1)^k a_k$ konvergiert, wenn $a_k$ eine konvergierende Nullfolge, also gegen 0 konvergiert.        
            \subsubsection{Majoranten- und Minoranten-Kriterium}
                Wenn es eine konvergente Reihe $\sum_{k=0}^{\infty}b_k$ gibt und $a_k \le b_k$ für alle $k \ge k_0$ gilt, dann ist die Reihe $\sum_{k=0}^{\infty}a_k$ ebenfalls eine konvergente Reihe.
                \\ \\
                Wenn es eine divergente Reihe $\sum_{k=0}^{\infty}b_k$ gibt,  und $|a_k| \ge b_k \ge 0$ gilt, dann ist die Reihe $\sum_{k=0}^{\infty}$ ebenfalls eine divergente Reihe.
                
                $\frac{1}{2}$ ist divergent.
                
                $\frac{1}{k^x}$ ist konvergent.
            \subsubsection{Quotientenkriterium}
                Wenn es eine Zahl $q < 1$ und ein $k_0 \ge \mathbb{N}_0$ gibt, so dass $\left|\frac{a_{k+1}}{a_k} \right| \le q$ für alle $k > k_0.$ \\
                Wenn unter den gleichen Vorraussetzungen $\left|\frac{a_{k+1}}{a_k} \right| > 1$ gilt, dann ist die Reihe divergent.
    \section{Differenzialrechnung}
    	\subsection{Differenzierbarkeit}
                  Eine Funktion $f(x)$ ist differenzierbar an der Stelle $x*$, wenn der Grenzwert von $\limFromTo{x*}{x} \frac{f(x) - f(x*)}{x - x*}$ existiert.
    	\subsection{typische Ableitungen}
    	   \begin{multicols}{2}		
    	   		$(x)' = 1$ \\
    			$(ax)' = a$ \\
    			$(ax^2)' = 2ax$ \\
    			$(\frac{1}{x})' = -\frac{1}{x^2}$ \\
    			$(\sqrt[]{x})' = \frac{1}{2\sqrt[]{x}}$ \\
    			$(ax^b)' = abx^{b-1}$ \\
    			$(e^x)' = e^x $ \\
    			$(a^x)' = a^x*log(a) $ \\
    			$ln(x)' = \frac{1}{x}$ \\
    		 	$(\sin x) = \cos x$ \\
    		 	$(\cos x) = -\sin x$ \\
    		 	$(\tan x) = \frac{1}{(\cos x)^2}$ \\ 
    		 	\end{multicols}
      \subsection{Verknüpfungsfunktionen}
      			Summenregel: $(f(x) + g(x))' = f(x)' + g(x)' $ \\ 
      			Produktregel: $(f(x)g(x))' = f(x)'g(x)+g(x)'f(x) $ \\
      			Quotientenregel: $(\frac{f(x)}{g(x)})' = \frac{f(x)'g(x)-g(x)'f(x)}{g(x)^2}$ \\
      			Kettenregel: 
      			$(f(g(x)))' = f(g(x))'g(x)'$ \\
     \section{Funktionen}
             \subsection{Monotonie}
                 Wenn $f(x) > f(y)$ für alle $x > y$ gilt, dann ist die Funktion monoton wachsend. \\
                 Wenn $f(x) < f(y)$ für alle $x < y$ gilt, dann ist die Funktion monoton fallend. \\
                 Wenn eine Bedingung nur "$\le$" oder "$\ge$" ist, dann fällt das monoton weg.
                 
             \subsection{Ungerade und gerade Funktionen}
                 Wenn gilt, dass $f(x) = f(-x)$ ist, dann ist die Funktion gerade. \\
                 Daraus folgt, dass die Funktion achsensymetrisch zur y-Achse ist.\\
                 Wenn gilt, dass $f(-x) = -f(x)$ ist, dann ist die Funktion ungerade.\\
                 Daraus folgt, dass die Funktion punktsymetrisch zum Ursprung ist.
              
              \subsection{Grenzwerte}
                  Grenzwerte von Funktionen werden über den $\lim$ gebildet.
                  Rechtsseitiger Grenzwert wird über die rechte Seite angenährt. \\
                  Notierung: $\limFromTo{x*^+}{x}$ \\ 
                  \\
                  Linksseitiger Grenzwert wird über die linke Seite angenährt. \\
                  Notierung: $\limFromTo{x*^-}{x}$ \\ 
                  \\
                  Asymptotisches Verhalten wird als Grenzwert einer Funktion gegen $\pm \infty$ definiert.
              \subsection{Stetigkeit}
                  Eine Funktion ist an der Stelle $x*$ stetig, wenn gilt:\\
                  $\limFromTo{x*+}{x} f(x) = \limFromTo{x*-}{x} f(x) = f(x*)$
                  \\ \\
                  Wird eine stetige Funktion mit einer anderen stetigen Funktion addiert, subtrahiert, multipliziert oder diviert, dann ist die resultierende Funktion ebenfalls stetig.\\
                  Wenn eine stetige Funktion vervielfacht wird, dann ist die resultierende Funktion wieder stetig. \\
                  \\
                  Jedes Polynom ist stetig und alle rationalen Funktionen in ihrem Definitionsbereich sind stetig.
                                    
              \subsection{Kurvendiskussion}
              \noindent
             	  Schnittpunkt x-Achse: $f(x) = 0$ \\
              	  Schnittpunkt y-Achse:  $f(0) = ...$\\
              	  $\bullet$Def.bereich z.B. $f(x)= \frac{1}{x} \rightarrow D_f={\rm I\!R}\backslash \{0\}$\\
              	  $\bullet$Nullstellen: pq-Formel (bei 3 unbek. Raten.)\\
                  $\bullet$\ Extremstellen: $f'(x) = 0$  \\
                  $f''(x_i) > 0$ $\rightarrow$ Tiefpunkt.\\
                  $f''(x_i) < 0$ $\rightarrow$ Hochpunkt. \\
                  Wendepunkt: $f''(x) = 0$ und $f'''(x_i) \neq 0$. \\
                  Sattelpunkt: $f''(x) = 0$ und $f'''(x) = 0$. \\
                  $\bullet$Verhalten im Unendlichen $\rightarrow \lim$ für + und - Unendlich.\\
                  Monoton steigend: $f'(x) > 0$ \\
                  Monoton fallend: $f'(x) < 0$ \\
                  Konkav: $f''(x) < 0$.\\
                  Konvex: $f''(x) > 0$.\\
                  Positiv Definit: $\forall\lambda>0 \rightarrow$ Minimum\\
                  Negativ Definit: $\forall\lambda<0 \rightarrow$ Maximum\\
                  indefinit $\rightarrow$ Sattelpunkt\\
                  semidefinit $\rightarrow$ B um A\\
              
                  
                  
                  
                  
                   	 
    \section{Taylor-Reihe}
        Die allgemeine Form der Taylor-Reihe einer Funktion $f(x)$, welche mindestens n-mal differenzierbar ist, ist: \\
        $T_n(x, x_e) = \sum_{k=0}^{n} \frac{f^{(k)}(x_e)}{k!}(x - x_e)^k$.
        \\ \\
        $T_1(x)$ ist die Tangente an der Stelle $x*$.
        Das Restglied $R_n$ wird durch Differenz der Funktion und dem Taylorpolynom gebildet. $R_n(x) = f(x) - T_n(x)$
        $R_n(x,x_ne) = \frac{f^{n+1}(c)}{(n+1)!}*(x-x_e)^{n+1} $
        
    \section{Integralrechnung}
    	$\int_a^b f(x)dx = [F(x)]_a^b = F(b) -F(a)$\\
    	$F'(x)=f(x)$
    	
    	
        $e^{Foo}$ u.ä. muss vorher substituiert werden!\\
        $\begin{matrix}
        \text{Funktion} & \text{Aufleitung} \\
        c & c \cdot x \\
        x^a, a \neq -1 & \frac{x^{a+1}}{a+1}\\
        x^{-1}, x \neq 0 & ln(|x|)\\
        e^x & e^x \\
        a^x & \frac{a^x}{ln(a)} \\
        sin(x) & -cos(x)\\
        cos(x) & sin(x)
        \end{matrix}$
        
        
        \subsection{Partielle Integration}
        Wenn $u$ und $v$ zwei differenzierbare Funktionen sind, dann gilt: \\
        $\int u' * v = (u * v) - \int u * v'$
        \subsection{Substitutionsregel}
        $\int f(g(x)) * g'(x) dx = \int f(y) dy$
        \begin{align}
            \int \frac{1}{5x - 7} dx &= ?\\
            z &= 5x - 7 \\
            \frac{dz}{dx} &= 5 \\   
            \frac{dz}{5} &= dx  \\
            \int \frac{1 * dz}{z * 5} &= \frac{1}{5} \int \frac{1}{z} dz \\
                                      &= \frac{1}{5} ln(z) \\
                                      &= \frac{1}{5} ln(5x-7)
        \end{align}
    \section{Multidimensionale Funktionen}
        Eine multidimensionale Funktion ist eine Funktion mit mehr als einer Variable.\\
       $f(x,y) = x^2 + 3xy$  
       
       \subsection{Partielle Ableitung}
       Eine partielle Ableitung ist die Ableitung zu einer Variablen.\\
       $f_x(x,y) = 2x + 3y$ \\
       $f_y(x,y) = 3$ \\
       Zweite partielle Ableitung: \\
       $f_{xx}(x,y) = 2$ , 
       $f_{xy}(x,y) = 3$ \\
       $f_{yx}(x,y) = 0$ ,
       $f_{yy}(x,y) = 0$ 
       \subsection{Gradient}
       Gradient besteht aus der ersten Ableitung.
       $grad_f(x,y) = ((2x + 3y), (3))$ \\
       Man kann eine Stelle einsetzen für x und y:\\
       $grad_f(3,2)= (12, 3)$.
              
       \subsection{Extremstellen mehrdimen. Funkt.}
       Zum Bestimmen der Extremstellen ist das Aufstellen der Hesse-Matrix notwendig.
       Dazu sind die Stellen, an den der Gradient 0 wird, die sogenannten kritischen Stellen. Wir setzen die kritischen Stellen in die Hesse-Matrix ein und bestimmen die Bestimmtheit der Matrix. Dazu bilden wir die Eigenwerte (über Determinante).\\
       
       \noindent
       $(\det>0) \land (W_{xx} \lor W_{yy})<0 \rightarrow$ Max.\\
       $(\det>0) \land (W_{xx} \lor W_{yy})>0 \rightarrow$ Min.\\
       $\det<0 \rightarrow$ Sattelp.\\
       $\det=0 \rightarrow$ k.A.
       
       
              \subsection{Hesse-Matrix - tolles Ding..}
       $H_f(x) = \left( 
       \begin{matrix}
       f_{xx}(x) & f_{xy}(x) \\
       f_{yx}(x) & f_{yy}(x)
       \end{matrix}
       \right)
       $
       
    \end{small}
    \end{multicols}
	\begin{small}
	\begin{multicols}{2}
		\section{Sin. Cos. Tan. Tabelle}
       \begin{tabular}{l | c  c  c  c  c  c  c  c  c  c }
       	\noindent
       	$x$ & $0$ & $\frac{1}{6}\pi$ & $\frac{1}{4}\pi$ & $\frac{1}{3}\pi$ & $\frac{1}{2}\pi$ & $\frac{2}{3}\pi$ & $\frac{3}{4}\pi$ & $\frac{5}{6}\pi$ & $\pi$ & $\frac{7}{6}\pi$\\\hline
       	$Grad$ & $0$ & $30$ & $45$ & $60$ & $90$ & $120$ & $135$ & $150$ & $180$ & $210$\\\hline
       	$\sin$ & $0$ & $\frac{1}{2}$ & $\frac{\sqrt{2}}{2}$ & $\frac{\sqrt{3}}{2}$ & $1$ & $\frac{\sqrt{3}}{2}$ & $\frac{\sqrt{2}}{2}$ & $\frac{1}{2}$ & $0$ & $-\frac{1}{2}$\\\hline
       	$\cos$ & $1$ & $\frac{\sqrt{3}}{2}$ & $\frac{\sqrt{2}}{2}$ & $\frac{1}{2}$ & $0$ & $-\frac{1}{2}$ & $-\frac{\sqrt{2}}{2}$ & $-\frac{\sqrt{3}}{2}$ & $-1$ & $-\frac{\sqrt{3}}{2}$\\\hline
       	$\tan$ & $0$ & $\frac{\sqrt{3}}{3}$ & $1$ & $\sqrt{3}$ &$\pm\infty$ & $-\sqrt{3}$ & $-1$ & $-\frac{\sqrt{3}}{3}$ & $0$ & $\frac{\sqrt{3}}{3}$
       	
       	
       \end{tabular}
    
    \end{multicols}
\end{small}
\end{document}
